\section{Organisation: Directories}
\label{sec:introduction:folders}

The \acs{template} template assumes a simple directory structure.
It has a main directory that holds all files relevant to the template itself.
Three complementary directories hold your texts, figures and appendices.



\subsection{Main Directory}
\label{sec:introduction:folders:main}
Your main directory contains the four sub-directories
\begin{enumerate}
	\item \texttt{chapters},
	\item \texttt{figures},
	\item \texttt{pretext}, and
	\item \texttt{appendices}.
\end{enumerate}
These directories are further described in separate sections. The main
directory also contains a couple of files: 
\begin{description}
	\item[\texttt{abstract.tex}] This file holds the abstract of your
	thesis.

	\item[\texttt{appendix.tex}] This file groups your appendices together.
	The file is \verb+\input+ into the main file \mbox{thesis.tex}.
	The appendices themselves should go into the \texttt{appendices}
	folder, and be included into \mbox{appendix.tex} via the \verb+\input+
	command.

	\item[\texttt{body.tex}] This file contains the main text of your
	thesis.
	The file is \verb+\input+ into the main file \mbox{thesis.tex}.
	The texts themselves should go into the \texttt{chapters} folder, and be
	included into \mbox{body.tex} via the \verb+\input+ command.

	\item[\texttt{commands.tex}] Your custom \LaTeX{} commands should go
	into this file.
	The file is \verb+\input+ into the preamble of the main file
	\mbox{thesis.tex}.
	By default, \mbox{commands.tex} defines commands for common
	abbreviations, and some environments.
	See Section~\ref{sec:packages:commands} for further information.

	\item[\texttt{literature.bib}] Your bibliographic references should go
	into this file.
	See Section~\ref{sec:guide:bibliographies} for further information.

	\item[\texttt{Makefile}] This file can make your life easier. It
	provides means for simple and complex compilation processes. For
	details, see Section~\ref{sec:makefiles:main}.

	\item[\texttt{packages.tex}] This is where you include \LaTeX{}
	packages.
	The file is \verb+\input+ as the first file into the preamble of the
	main file \mbox{thesis.tex}.
	By default, it loads the auxiliary packages as described in
	Section~\ref{sec:packages:auxiliary_packages}.

	\item[\texttt{thesis.tex}] This is the main file for your thesis.
	It defines the document class, includes the preamble, and loads files
	\mbox{body.tex} and \mbox{appendix.tex}.
	It also sets up the title page, legalities, and the table of contents.

	\item[\texttt{upb\_cs\_thesis.cls}] This is defines the document class
	used by the \ac{template} template.
	See Chapter~\ref{ch:class_file} for details.
\end{description}



\subsection{Chapters Directory}
\label{sec:introduction:folders:chapters}
This directory is where the \mbox{.tex} files of your thesis go.
It is advisable to have one file per chapter.
For long chapters, you should create additional files for smaller bits of text,
\eg{} on the section or even subsection level.
The low level files should then be \verb+\input+ into the respective chapter's
\mbox{.tex} file, while the chapter's \mbox{.tex} file should be \verb+input+
in main directory's \mbox{body.tex}.



\subsection{Figures Directory}
\label{sec:introduction:folders:figures}
This directory is where your graphics go.
See Section~\ref{sec:guide:graphics} on how to include graphics in your thesis.
By default, this directory contains the university's logo
(\texttt{uni-logo.pdf}), a \texttt{Makefile}, and the \texttt{template\_files}
sub-directory. 

The university's logo is used on the title page.
The \emph{Makefile} of the \emph{figures} directory converts some types of raw
data into graphics; see Section~\ref{sec:makefiles:figures} for details.
Finally, the \emph{template\_files} sub-directory contains the
\texttt{tikz\_template.tex} file, that can be used to create
tikz\footnote{\TeX{} Ist Kein Zeichenprogramm, a \LaTeX{} package for
programming graphics} graphics that the figures directory's \emph{Makefile} can
properly process.



\subsection{Pretext Directory}
\label{sec:introduction:folders:pretext}
This directory contains two files \texttt{erklaerung.tex} and
\texttt{titlepage.tex}.
The former contains a legal statement, the latter defines the layout of your
titlepage.
Typically, you do not need to edit these files.
You fill in the details of your title page as described in
Section~\ref{sec:introduction:start}.



\subsection{Appendices Directory}
\label{sec:introduction:folders:appendices}
This directory is where your thesis's appendices go.
The files should be \verb+\input+ into your thesis via the \mbox{appendix.tex}
file.
By default, this directory is empty.
